%%%%%%%%%%%%%%%%%
% This is an sample CV template created using altacv.cls
% (v1.2, 11 February 2020) written by LianTze Lim (liantze@gmail.com). Now compiles with pdfLaTeX, XeLaTeX and LuaLaTeX.
%
%% It may be distributed and/or modified under the
%% conditions of the LaTeX Project Public License, either version 1.3
%% of this license or (at your option) any later version.
%% The latest version of this license is in
%%    http://www.latex-project.org/lppl.txt
%% and version 1.3 or later is part of all distributions of LaTeX
%% version 2003/12/01 or later.
%%%%%%%%%%%%%%%%

%% If you need to pass whatever options to xcolor
\PassOptionsToPackage{dvipsnames}{xcolor}

%% If you are using \orcid or academicons
%% icons, make sure you have the academicons
%% option here, and compile with XeLaTeX
%% or LuaLaTeX.
% \documentclass[10pt,a4paper,academicons]{altacv}

%% Use the "normalphoto" option if you want a normal photo instead of cropped to a circle
% \documentclass[10pt,a4paper,normalphoto]{altacv}

\documentclass[10pt,a4paper,ragged2e]{altacv}

%% AltaCV uses the fontawesome and academicon fonts
%% and packages.
%% See http://texdoc.net/pkg/fontawesome and http://texdoc.net/pkg/academicons for full list of symbols. You MUST compile with XeLaTeX or LuaLaTeX if you want to use academicons.

% Change the page layout if you need to
\geometry{left=1.25cm,right=1.25cm,top=0.8cm,bottom=1cm,columnsep=1cm}

% The paracol package lets you typeset columns of text in parallel
\usepackage{paracol}
\usepackage{ragged2e}
% Change the font if you want to, depending on whether
% you're using pdflatex or xelatex/lualatex
\ifxetexorluatex
  % If using xelatex or lualatex:
  \setmainfont{Lato}
\else
  % If using pdflatex:
  \usepackage[utf8]{inputenc}
  \usepackage[T1]{fontenc}
  \usepackage[default]{lato}
\fi

% Change the colours if you want to
\definecolor{Mulberry}{HTML}{72243D}
\definecolor{SlateGrey}{HTML}{2E2E2E}
\definecolor{LightGrey}{HTML}{666666}
\colorlet{heading}{Sepia}
\colorlet{accent}{Mulberry}
\colorlet{emphasis}{SlateGrey}
\colorlet{body}{LightGrey}
\usepackage{multicol}

% Change the bullets for itemize and rating marker
% for \cvskill if you want to
\renewcommand{\itemmarker}{{\small\textbullet}}
\renewcommand{\ratingmarker}{\faCircle}

%% sample.bib contains your publications
\addbibresource{sample.bib}

\begin{document}

\name{Arianna Cantarella}
%\tagline{Your Position or Tagline Here}
%% You can add multiple photos on the left or right
\photoR{3.2cm}{D}
% \photoL{2.5cm}{Yacht_High,Suitcase_High}

\personalinfo{%
  % Not all of these are required!
  % You can add your own with \printinfo{symbol}{detail}
  
\printinfo{\large{Date of birth}}{\large{\normalfont{21/04/1998}}} \hspace{8mm}
  %\vspace{2mm}
   %\phone{\large{\normalfont{+39 3472832818}}}
  \vspace{2mm}
  \mailaddress{\large{\normalfont{arianna.cantarella@unipr.it}}}
  \mailaddress{\large{\normalfont{ariannacantarella@gmail.com}}}
  \vspace{2mm}
  %\github{\large{\normalfont{github.com/AriannaCantarella}}}
%\vspace{2mm}
  \location{\large{\normalfont{Parma, Italy}}}
  %\homepage{www.homepage.com/}
  %\twitter{@twitterhandle}
  %\linkedin{linkedin.com/in/yourid}
  
  %% You MUST add the academicons option to \documentclass, then compile with LuaLaTeX or XeLaTeX, if you want to use \orcid or other academicons commands.
  % \orcid{orcid.org/0000-0000-0000-0000}
}

\makecvheader

%% Depending on your tastes, you may want to make fonts of itemize environments slightly smaller
% \AtBeginEnvironment{itemize}{\small}

%% Set the left/right column width ratio to 6:4.
%\columnratio{0.6}

% Start a 2-column paracol. Both the left and right columns will automatically
% break across pages if things get too long.
\begin{paracol}{1}
\vspace{-7mm}
\cvsection{Education and training}
\cvevent{\textbf{PhD student in Condensed Matter Physics}}{University of Parma}{Nov 2023 - ongoing}{Parma, Italy}
Currently, my research focuses on first-principles calculations to unravel the electronic structures of organic compounds. This work contributes to the study of chiral molecules, a key aspect within the ERC Synergy project CASTLE.\\
\vspace{1mm}
\textbf{Supervisors:} Prof. Stefano Carretta (UniPR), Prof. Pietro Bonfà (UniPR).\\
\divider


\cvevent{\textbf{Master's Degree in Condensed Matter Physics}
\newline \small{110/110 cum laude}}{University of Parma}{Oct 2020 - Apr 2023}{Parma, Italy}
%\newline Oriented to theoretical models and numerical simulations to understand properties of states of matter.\\
\vspace{2mm}
\textbf{Thesis title:} high-throughput detection and restoring of missing hydrogens in inorganic crystal structure databases;\\
conducted at École polytechnique fédérale de Lausanne (EPFL).
\\
\vspace{1mm}
\textbf{Supervisors:} Prof. Pietro Bonfà (UniPR), Dott. Giovanni Pizzi (EPFL, PSI), Dott. Marnik Bercx (EPFL).

\divider

\cvevent{\textbf{Bachelor's Degree in Physics}
\newline \small{100/110}}{University of Parma}{Sept 2017-Oct 2020}{Parma, Italy}
\textbf{Thesis title:} simulations and tests of a device based on Hesuler Alloys for the conversion of thermomagnetic energy into electrical
energy.\\
\vspace{1mm}
\textbf{Supervisors:} Prof. Massimo Solzi (UniPR), Prof. Francesco Cugini (UniPR).


% \cvevent{High School Diploma
% \newline \small{90/100}}{Liceo Scientifico G. Marconi}{2012-2017}{Parma, Italy}
%\divider

\cvsection{Technical skills and interests}
\vspace{-6mm}
\begin{multicols}{2}
\normalfont{Python for materials science (pymatgen, ASE)
\vspace{2mm}
\newline MATLAB for numerical simulations
\vspace{2mm}
\newline Q-Chem for DFT and TDDFT calculations
\vspace{2mm}
\newline AiiDA Python-based infrastructure for automated workflows in computational science
\vspace{2mm}
\newline Quantum ESPRESSO for DFT calculations
\vspace{2mm}
\newline Experience in utilizing High-Performance Computing (HPC) systems for running simulations}
\vspace{2mm}
\newline \textbf{Interests}: simulations in materials, solid state physics, computational physics, computational chemistry
\end{multicols}

\vspace{-5mm}
\begin{minipage}[t]{0.55\textwidth}
\cvsection{Awards}
\cvevent{\normalfont{INSPIRE Potentials – MARVEL Master's Fellowship}}{EPFL}{Oct 2022 - Mar 2023}{Lausanne, Switzerland}
I conducted my Master's research
thesis in the Theory and Simulation of Materials Laboratory group, led by Prof. Nicola Marzari.
\end{minipage}
\hfill
\begin{minipage}[t]{0.38\textwidth}
\cvsection{Language Skills}
%\vspace{-0.5mm}
\cvskill{\large{Italian } \normalfont{(mother tongue)} }{5}
\vspace{3mm}
\cvskill{\large{English} }{4}
\vspace{3mm}
\cvskill{\large {French}}{2}
\end{minipage}

\begin{minipage}{0.55\textwidth}
    \cvsection{Work Experience}
  \vspace{-2mm} % Adjust vertical space if necessary
  \cvevent{Tutor Fisica della Materia}{}{03/2024-06/2024}{University of Parma, Italy}

  \cvevent{Valorisation of Master's Degree}{}{May 2023 - Aug 2023}{EPFL, Lausanne, Switzerland}

  % \cvevent{STEM tutor}{University of Parma}{02/2021-03/2021}  {Parma, Italy}

\end{minipage}
\hfill
\begin{minipage}{0.4\textwidth}
    \cvsection{Conferences}
\cvevent{MARVEL Review and Retreat}{}{January 18-20 2023}{Grindelwald, Switzerland}

\cvevent{International Conference on Magnetism}{}{Jun 30 - Jul 5 2024}{Bologna, Italy}


\end{minipage}

















%\medskip
%\divider
%\newpage




% \vspace{8mm}
% \begin{figure}[h!]
% \begin{minipage}[b]{0.3\textwidth}
% \centering
% \Large{Parma, 12/09/2023}
% \end{minipage}
% \hfill
% \begin{minipage}[b]{0.5\textwidth}
% \vspace{5mm}
% \centering
% \includegraphics[scale=0.1]{Firma.png}

% \end{minipage}

% \end{figure}
\end{paracol}


\end{document}
